\chapter{Dnevnik promjena dokumentacije}
		
		%\textbf{\textit{Kontinuirano osvježavanje}}\\
				
		
		\begin{longtblr}[
				label=none
			]{
				width = \textwidth, 
				colspec={|X[2]|X[13]|X[3]|X[3]|}, 
				rowhead = 1
			}
			\hline
			\textbf{Rev.}	& \textbf{Opis promjene/dodatka} & \textbf{Autori} & \textbf{Datum}\\[3pt] \hline
			0.1 & Napravljen predložak.	& Jura Starčević & 5.11.2022. 		\\[3pt] \hline 
			0.2	& Dodan opis & Jura Starčević & 6.11.2022. 	\\[3pt] \hline 
			0.3 & Dodano i djelomično uređeno poglavlje 3.1 & Marko Štrk & 8.11.2022. \\[3pt] \hline 
			0.4 & Početak dodavanja informacija o bazi & Jura Starčević & 15.11.2022. \\[3pt] \hline 
			0.5 & Dodan djelomičan dijagram obrazaca uporabe & Karla Udiljak & 15.11.2022. \\[3pt] \hline 
			0.6 & Dodani i razrađeni Use Caseovi & Marko Štrk & 16.11.2022. \\[3pt] \hline 
			0.7 & Dodavanje arhitekture i opisa baze podataka & Fran Markulin & 16.11.2022. \\[3pt] \hline 
			0.8 & Uređivanje opisa i arhitekture & Jura Starčević & 16.11.2022. \\[3pt] \hline 
			0.9 & Dodavanje potpunog dijagrama obrazaca uporabe, sekvencijskih dijagrama te njihovih opisa & Karla Udiljak & 16.11.2022. \\[3pt] \hline 
			0.10 & Dodani Ostali zahtjevi i UML dijagram & Marko Štrk & 17.11.2022. \\[3pt] \hline 
			\textbf{1.0} & Finalna verzija za prvi ciklus & Jura Starčević & 18.11.2022. \\[3pt] \hline 
			
		\end{longtblr}
	
	
		%\textit{Moraju postojati glavne revizije dokumenata 1.0 i 2.0 na kraju prvog i drugog ciklusa. Između tih revizija mogu postojati manje revizije već prema tome kako se dokument bude nadopunjavao. Očekuje se da nakon svake značajnije promjene (dodatka, izmjene, uklanjanja dijelova teksta i popratnih grafičkih sadržaja) dokumenta se to zabilježi kao revizija. Npr., revizije unutar prvog ciklusa će imati oznake 0.1, 0.2, …, 0.9, 0.10, 0.11.. sve do konačne revizije prvog ciklusa 1.0. U drugom ciklusu se nastavlja s revizijama 1.1, 1.2, itd.}