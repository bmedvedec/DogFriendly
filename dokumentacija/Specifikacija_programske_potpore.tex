\chapter{Specifikacija programske potpore}
		
	\section{Funkcionalni zahtjevi}
			
			%\textbf{\textit{dio 1. revizije}}\\
			
			%\textit{Navesti \textbf{dionike} koji imaju \textbf{interes u ovom sustavu} ili  \textbf{su nositelji odgovornosti}. To su prije svega korisnici, ali i administratori sustava, naručitelji, razvojni tim.}\\
				
			%\textit{Navesti \textbf{aktore} koji izravno \textbf{koriste} ili \textbf{komuniciraju sa sustavom}. Oni mogu imati inicijatorsku ulogu, tj. započinju određene procese u sustavu ili samo sudioničku ulogu, tj. obavljaju određeni posao. Za svakog aktora navesti funkcionalne zahtjeve koji se na njega odnose.}\\
			
			
			\noindent \textbf{Dionici:}
			
			\begin{packed_enum}
				
				\item Vlasnik (naručitelj)
				\item Vlasnik psa
				\item Vlasnik obrta
				\item Razvojni tim
				
			\end{packed_enum}
			
			\noindent \textbf{Aktori i njihovi funkcionalni zahtjevi:}
			
			
			\begin{packed_enum}
				\item  \underbar{Neregistrirani/neprijavljeni korisnik može:}
				
				\begin{packed_enum}
					
					\item pregledati lokacije na karti
					\item odabrati lokaciju i dobiti prikaz opcih informacija (ime lokacije, adresa, telefon, OIB, kratak opis, djelatnost)
					\item se registrirati u sustav, stvoriti novi korisnicki račun za koji su mu potrebni korisničko ime, lozinka, ime, prezime, broj mobitela, e-mail adresa
					
					
				\end{packed_enum}
			
				\item  \underbar{Registirani korisnik može:}
				
				\begin{packed_enum}
					
					\item pregledavati i mijenjati osobne podatke
					\item izbrisati svoj korisnički račun
					\item pisati recenzije i dati ocjene
					\item pregledati recenzije
					
				\end{packed_enum}
				
			
				\item  \underbar{Baza podataka(sudionik):}
				
				\begin{packed_enum}
					
					\item pohranjuje sve podatke o korisnicima i njihovim ovlastima
					\item pohranjuje sve podatke o lokacijama

					
				\end{packed_enum}
				
			\end{packed_enum}
			
			\eject 
			
			
				
			\subsection{Obrasci uporabe}
				
				%\textbf{\textit{dio 1. revizije}}
				
				%\subsubsection{Opis obrazaca uporabe}
					%\textit{Funkcionalne zahtjeve razraditi u obliku obrazaca uporabe. Svaki obrazac je potrebno razraditi prema donjem predlošku. Ukoliko u nekom koraku može doći do odstupanja, potrebno je to odstupanje opisati i po mogućnosti ponuditi rješenje kojim bi se tijek obrasca vratio na osnovni tijek.}\\
					

					\noindent \underbar{\textbf{UC1 - Pregled lokacija na karti}}
					\begin{packed_item}
	
						\item \textbf{Glavni sudionik: }Registrirani korisnik, Neregistrirani korisnik
						\item  \textbf{Cilj:} Pregledati lokacije i osnovne informacije
						\item  \textbf{Sudionici:} Baza podataka, Google Maps API
						\item  \textbf{Preduvjet:} -
						\item  \textbf{Opis osnovnog tijeka:}
						
						\item[] \begin{packed_enum}
	
							\item Karta je prikazana prilikom učitavanja aplikacije
							\item Korisnik na karti odabire lokaciju
							\item Prikazuju se informacije o lokaciji i ponudi

						\end{packed_enum}
						
					\end{packed_item}
					
					\noindent \underbar{\textbf{UC2 - Registracija  obrta}}
					\begin{packed_item}
	
						\item \textbf{Glavni sudionik: }Neregistrirani korisnik
						\item  \textbf{Cilj:} Stvoriti korisnički račun za pristup sustavu
						\item  \textbf{Sudionici:} Baza podataka
						\item  \textbf{Preduvjet:} -
						\item  \textbf{Opis osnovnog tijeka:}
						
						\item[] \begin{packed_enum}
	
							\item Korisnik odabire opciju za registraciju
							\item Korisnik odabire opciju "Company"
							\item Korisnik unosi podatke
							\item Korisnik prima e-mail za potvrdu registracije
							\item Korisnik potvrđuje registraciju
							\item Prikazuje se poruka o uspješnoj registraciji

						\end{packed_enum}
						
						\item  \textbf{Opis mogućih odstupanja:}
						
						\item[] \begin{packed_item}
	
							\item[2.a] Odabir već zauzetog korisničkog imena i/ili e-maila, unos korisničkih podataka u nedozvoljenom formatu ili pružanje neispravnog e-maila, ne zadovoljavanje kompleksnosti lozinke
							\item[] \begin{packed_enum}
								
								\item Sustav obavještava korisnika o neuspješnom upisu i vraća ga na stranicu za registraciju
								\item Korisnik mijenja podatke te završava unos ili odustaje od registracije
								
							\end{packed_enum}
							
						\end{packed_item}
					\end{packed_item}
					
					\noindent \underbar{\textbf{UC3 - Registracija privatne osobe}}
					\begin{packed_item}
	
						\item \textbf{Glavni sudionik: }Neregistrirani korisnik
						\item  \textbf{Cilj:} Stvoriti korisnički račun za pristup sustavu
						\item  \textbf{Sudionici:} Baza podataka
						\item  \textbf{Preduvjet:} -
						\item  \textbf{Opis osnovnog tijeka:}
						
						\item[] \begin{packed_enum}
	
							\item Korisnik odabire opciju za registraciju
							\item Korisnik odabire opciju "personal"
							\item Korisnik unosi podatke
							\item Korisnik prima e-mail za potvrdu registracije
							\item Korisnik potvrđuje registraciju
							\item Prikazuje se poruka o uspješnoj registraciji

						\end{packed_enum}
						
						\item  \textbf{Opis mogućih odstupanja:}
						
						\item[] \begin{packed_item}
	
							\item[2.a] Odabir već zauzetog korisničkog imena i/ili e-maila, unos korisničkih podataka u nedozvoljenom formatu ili pružanje neispravnog e-maila, ne zadovoljavanje kompleksnosti lozinke
							\item[] \begin{packed_enum}
								
								\item Sustav obavještava korisnika o neuspjelom upisu i vraća ga na stranicu za registraciju
								\item Korisnik mijenja podatke te završava unos ili odustaje od registracije
								
							\end{packed_enum}
							
						\end{packed_item}
					\end{packed_item}
					
					\noindent \underbar{\textbf{UC4 - Prijava u sustav}}
					\begin{packed_item}
	
						\item \textbf{Glavni sudionik: } Registrirani korisnik
						\item  \textbf{Cilj:} Dobiti pristup korisničkom sučelju
						\item  \textbf{Sudionici:} Baza podataka
						\item  \textbf{Preduvjet:} Registracija
						\item  \textbf{Opis osnovnog tijeka:}
						
						\item[] \begin{packed_enum}
	
							\item Unos e-maila i lozinke
							\item Potvrda ispravnosti unesenih podataka
							\item Pristup korisničkim funkcijama

						\end{packed_enum}
						
						\item  \textbf{Opis mogućih odstupanja:}
						
						\item[] \begin{packed_item}
	
							\item[2.a] Neispravno korisničko ime/lozinka
							\item[] \begin{packed_enum}
								
								\item Sustav obavještava korisnika o neuspjelom upisu i vraća ga na stranicu za prijavu
								
							\end{packed_enum}

						\end{packed_item}
					\end{packed_item}
					
					\noindent \underbar{\textbf{U5 - Pregled osobnih podataka}}
					\begin{packed_item}
	
						\item \textbf{Glavni sudionik: } Registrirani korisnik
						\item  \textbf{Cilj:} Pregledati osobne podatke
						\item  \textbf{Sudionici:} Baza podataka
						\item  \textbf{Preduvjet:} Korisnik je prijavljen
						\item  \textbf{Opis osnovnog tijeka:}
						
						\item[] \begin{packed_enum}
	
							\item Korisnik odabire opciju "Osobni podatci"
							\item Aplikacija prikazuje osobne podatke korisnika

						\end{packed_enum}
						
						
					\end{packed_item}
					
					\noindent \underbar{\textbf{UC6 - Promjena osobnih podataka}}
					\begin{packed_item}
	
						\item \textbf{Glavni sudionik: } Registrirani korisnik
						\item  \textbf{Cilj:} Promijeniti osobne podatke
						\item  \textbf{Sudionici:} Baza podataka
						\item  \textbf{Preduvjet:} Korisnik je prijavljen
						\item  \textbf{Opis osnovnog tijeka:}
						
						\item[] \begin{packed_enum}
	
							\item Korisnik odabire opciju za promjenu podataka
							\item Korisnik mijenja svoje osobne podatke
							\item Korisnik sprema promjene
							\item Baza podataka se ažurira

						\end{packed_enum}
						
						\item  \textbf{Opis mogućih odstupanja:}
						
						\item[] \begin{packed_item}
	
							\item[2.a] Korisnik promjeni svoje osobne podatke, ali ne odabere opciju "Spremi promjenu"
							\item[] \begin{packed_enum}
								
								\item Sustav obavještava korisnika da nije spremio podatke prije izlaska iz prozora

							\end{packed_enum}
							\item[2.b] Odabir već zauzetog korisničkog imena i/ili e-maila, unos korisničkog podatka u nedozvoljenom formatu ili pružanje neispravnog e-maila, ne zadovoljavanje kompleksnosti lozinke, promijenjena lozinka je jednaka trenutnoj
							\item[] \begin{packed_enum}
								
								\item Sustav obavještava korisnika o neuspjelom upisu i vraća ga na stranicu za promjenu podataka
								\item Korisnik mijenja podatke te završava unos ili odustaje od promjene podataka
								
							\end{packed_enum}
							
							
						\end{packed_item}
					\end{packed_item}
					
					\noindent \underbar{\textbf{UC7 - Brisanje korisničkog računa}}
					\begin{packed_item}
	
						\item \textbf{Glavni sudionik: } Registrirani korisnik
						\item  \textbf{Cilj:} Izbrisati svoj korisnički račun
						\item  \textbf{Sudionici:} Baza podataka
						\item  \textbf{Preduvjet:} Korisnik je prijavljen
						\item  \textbf{Opis osnovnog tijeka:}
						
						\item[] \begin{packed_enum}
	
							\item Otvara se stranica s osobnim podacima korisnika
							\item Korisnik briše račun
							\item Korisnički račun se izbriše iz baze podataka
							\item Otvara se stranica za registraciju

						\end{packed_enum}
						
						
					\end{packed_item}
					
					\noindent \underbar{\textbf{UC8 - Pregled informacija o lokaciji}}
					\begin{packed_item}
	
						\item \textbf{Glavni sudionik: }Registrirani/Neregistrirani korisnik
						\item  \textbf{Cilj:} Pregledati detaljnije informacije o lokaciji
						\item  \textbf{Sudionici:} Baza podataka, Google Maps API
						\item  \textbf{Preduvjet:} -
						\item  \textbf{Opis osnovnog tijeka:}
						
						\item[] \begin{packed_enum}
	
							\item Korisnik odabire određenu lokaciju
							\item Prikazuju se detaljne informacije o lokaciji

						\end{packed_enum}
						
					\end{packed_item}
					
					\noindent \underbar{\textbf{UC9 - Pregled informacija o obrtu}}
					\begin{packed_item}
	
						\item \textbf{Glavni sudionik: }Registrirani/Neregistrirani korisnik
						\item  \textbf{Cilj:} Pregledati detaljnije informacije o obrtu
						\item  \textbf{Sudionici:} Baza podataka, Google Maps API
						\item  \textbf{Preduvjet:} -
						\item  \textbf{Opis osnovnog tijeka:}
						
						\item[] \begin{packed_enum}
	
							\item Korisnik odabire određeni obrt
							\item Prikazuju se detaljne informacije o tom obrtu

						\end{packed_enum}
						
					\end{packed_item}
					
					\noindent \underbar{\textbf{UC10 - Filtriranje lokacije prema kategoriji}}
					\begin{packed_item}
	
						\item \textbf{Glavni sudionik: }Registrirani/Neregistrirani korisnik
						\item  \textbf{Cilj:} Filtrirati prikazane lokacije
						\item  \textbf{Sudionici:} Baza podataka, Google Maps API
						\item  \textbf{Preduvjet:} -
						\item  \textbf{Opis osnovnog tijeka:}
						
						\item[] \begin{packed_enum}
	
							\item Karta je prikazana prilikom učitavanja aplikacije
							\item Korisnik odabire željene kategorije
							\item Prikazuju se samo lokacije koje odgovaraju unesenim kategorijama

						\end{packed_enum}
						
					\end{packed_item}
					
					\noindent \underbar{\textbf{UC11 - Filtriranje lokacije prema unosu u polje za pretraživanje}}
					\begin{packed_item}
	
						\item \textbf{Glavni sudionik: }Registrirani/Neregistrirani korisnik
						\item  \textbf{Cilj:} Filtrirati prikazane lokacije
						\item  \textbf{Sudionici:} Baza podataka, Google Maps API
						\item  \textbf{Preduvjet:} -
						\item  \textbf{Opis osnovnog tijeka:}
						
						\item[] \begin{packed_enum}
	
							\item Karta je prikazana prilikom učitavanja aplikacije
							\item Korisnik unosi željeni tekst u polje za pretraživanje
							\item Prikazuju se samo lokacije koje odgovaraju unesenom tekstu

						\end{packed_enum}
						
					\end{packed_item}
					
					\noindent \underbar{\textbf{UC12 - Dohvaćanje lokacije uređaja}}
					\begin{packed_item}
	
						\item \textbf{Glavni sudionik: }Registrirani/Neregistrirani korisnik
						\item  \textbf{Cilj:} Prikazati točnu lokaciju korisnika
						\item  \textbf{Sudionici:} Google Maps API
						\item  \textbf{Preduvjet:} Prihvaćena privola za dohvaćanje lokacije uređaja
						\item  \textbf{Opis osnovnog tijeka:}
						
						\item[] \begin{packed_enum}
	
							\item Karta je prikazana prilikom učitavanja aplikacije
							\item Traži se privola korisnika
							\item Prikazuje se točna lokacija uređaja

						\end{packed_enum}
						
						\item  \textbf{Opis mogućih odstupanja:}
						
						\item[] \begin{packed_item}
	
							\item[2.a] Korisnik ne daje privolu
							\item[] \begin{packed_enum}
								
								\item Ne prikazuje se korisnikova lokacija
								
							\end{packed_enum}
							
						\end{packed_item}
						
					\end{packed_item}
					
					\noindent \underbar{\textbf{UC13 - Privola za dohvaćanje lokacije uređaja}}
					\begin{packed_item}
	
						\item \textbf{Glavni sudionik: }Registrirani/Neregistrirani korisnik
						\item  \textbf{Cilj:} Dobiti od korisnika dozvolu za lociranje
						\item  \textbf{Sudionici:} Google Maps API
						\item  \textbf{Preduvjet:} -
						\item  \textbf{Opis osnovnog tijeka:}
						
						\item[] \begin{packed_enum}
	
							\item Prikazuje se prozor u kojem se od korisnika traži dopuštenje
							\item Korisnik daje dopuštenje
							\item Prikazuje se točna lokacija uređaja

						\end{packed_enum}
						
						\item  \textbf{Opis mogućih odstupanja:}
						
						\item[] \begin{packed_item}
	
							\item[2.a] Korisnik ne daje privolu
							\item[] \begin{packed_enum}
								
								\item Ne prikazuje se korisnikova lokacija
								
							\end{packed_enum}
							
						\end{packed_item}
						
					\end{packed_item}
					
					\noindent \underbar{\textbf{UC14 - Dodavanje lokacija}}
					\begin{packed_item}
	
						\item \textbf{Glavni sudionik: }Registrirani korisnik
						\item  \textbf{Cilj:} Dodati lokaciju u sustav
						\item  \textbf{Sudionici:} Baza podataka, Google Maps API
						\item  \textbf{Preduvjet:} Korisnik je prijavljen
						\item  \textbf{Opis osnovnog tijeka:}
						
						\item[] \begin{packed_enum}
	
							\item Korisnik odabire opciju "Dodaj lokaciju"
							\item Korisnik unosi podatke o lokaciji
							\item Podatci o lokaciji se spremaju u bazu podataka
							\item Lokacija se prikazuje na karti

						\end{packed_enum}
						
						\item  \textbf{Opis mogućih odstupanja:}
						
						\item[] \begin{packed_item}
	
							\item[2.a] Korisnik unosi podatke u nedozvoljenom formatu 
							\item[] \begin{packed_enum}
								
								\item Sustav obavještava korisnika o pogrešci 
								
							\end{packed_enum}
							
						\end{packed_item}
						
					\end{packed_item}
					
					\noindent \underbar{\textbf{UC15 - Ocjenjivanje prikladnosti lokacija}}
					\begin{packed_item}
	
						\item \textbf{Glavni sudionik: }Registrirani korisnik
						\item  \textbf{Cilj:} Ocijeniti je li lokacija prikladna za pse ili nije
						\item  \textbf{Sudionici:} Baza podataka, Google Maps API
						\item  \textbf{Preduvjet:} Lokacija postoji u bazi podataka
						\item  \textbf{Opis osnovnog tijeka:}
						
						\item[] \begin{packed_enum}
	
							\item Korisnik odabire opciju "Ocijeni lokaciju"
							\item Korisnik daje svoju ocjenu
							\item Baza podataka se ažurira

						\end{packed_enum}
						
					\end{packed_item}
					
					\noindent \underbar{\textbf{UC16 - Reklama obrta uz naknadu}}
					\begin{packed_item}
	
						\item \textbf{Glavni sudionik: }Registrirani korisnik
						\item  \textbf{Cilj:} Istaknuti svoj objekt na karti
						\item  \textbf{Sudionici:} Baza podataka, Google Maps API
						\item  \textbf{Preduvjet:} -
						\item  \textbf{Opis osnovnog tijeka:}
						
						\item[] \begin{packed_enum}
	
							\item Korisnik odabire opciju "Istakni obrt"
							\item Korisnik unosi kartične podatke
							\item Korisnik plaća naknadu
							\item Baza podataka se ažurira
							\item Lokacija postaje istaknuta na karti

						\end{packed_enum}
						
						\item  \textbf{Opis mogućih odstupanja:}
						
						\item[] \begin{packed_item}
	
							\item[3.a] Kartica je odbijena
							\item[] \begin{packed_enum}
								
								\item Prikazuje se poruka o grešci, korisnika se vraća na početnu stranicu
								
							\end{packed_enum}
							
						\end{packed_item}
						
					\end{packed_item}
					
					\noindent \underbar{\textbf{UC17 - Uređivanje podataka o obrtu}}
					\begin{packed_item}
	
						\item \textbf{Glavni sudionik: }Registrirani korisnik
						\item  \textbf{Cilj:} Prikazati točnu lokaciju korisnika
						\item  \textbf{Sudionici:} Baza podataka
						\item  \textbf{Preduvjet:} -
						\item  \textbf{Opis osnovnog tijeka:}
						
						\item[] \begin{packed_enum}
	
							\item Korisnik odabire opciju za promjenu podataka
							\item Korisnik mijenja podatke o svom obrtu
							\item Korisnik sprema promjene
							\item Baza podataka se ažurira

						\end{packed_enum}
						
						\item  \textbf{Opis mogućih odstupanja:}
						
						\item[] \begin{packed_item}
	
							\item[2.a] Korisnik unosi podatke u nedozvoljenom formatu
							\item[] \begin{packed_enum}
								
								\item Sustav obavještava korisnika o neuspjelom upisu
								
							\end{packed_enum}
							
						\end{packed_item}
						
					\end{packed_item}
					
					
				
					
				\subsubsection{Dijagrami obrazaca uporabe}
					
					%\textit{Prikazati odnos aktora i obrazaca uporabe odgovarajućim UML dijagramom. Nije nužno nacrtati sve na jednom dijagramu. Modelirati po razinama apstrakcije i skupovima srodnih funkcionalnosti.}
					\begin{figure}[H]
						\includegraphics[scale=0.45]{slike/UseCaseDiagram2.png}
						\centering
						\caption{Dijagram obrazaca uporabe}
						\label{fig:promjene}
					\end{figure}
				\eject		
				
			\subsection{Sekvencijski dijagrami}
				
				%\textbf{\textit{dio 1. revizije}}\\
				
				%\textit{Nacrtati sekvencijske dijagrame koji modeliraju najvažnije dijelove sustava (max. 4 dijagrama). Ukoliko postoji nedoumica oko odabira, razjasniti s asistentom. Uz svaki dijagram napisati detaljni opis dijagrama.}
				
				Obrazac uporabe UC1 - Pregled lokacija na karti

				Registrirani ili neregistrirani korisnik šalje zahtjev za prikaz karte s lokacijama. Poslužitelj dohvaća markere iz baze podataka.
				Zatim markere preuzima Google Maps API koji ih dodaje na kartu. Poslužitelj dohvaća kartu s markerima te je prikazuje korisniku.
				Dok god se korisnik kreće po karti, poslužitelj od njega prima zahtjeve za svaki korak jer pomake na karti mora zatražiti od 
				Google Maps API-ja čija je uloga omogućiti kretanje po karti. Poslužitelj zatim prikazuje pomake na karti korisniku. 
				
				\begin{figure}[H]
					\includegraphics[scale=0.63]{slike/SekvencijskiDijagram1.png}
					\centering
					\caption{Sekvencijski dijagram za UC1}
					\label{fig:promjene}
				\end{figure}

				Obrazac uporabe U14 - Dodavanje lokacija

				Registrirani korisnik šalje zahtjev za prikaz karte s lokacijama kako bi mogao odabrati opciju dodavanja lokacije.Poslužitelj dohvaća markere iz baze podataka.
				Zatim markere preuzima Google Maps API koji ih dodaje na kartu. Poslužitelj dohvaća kartu s markerima te je prikazuje korisniku. Korisnik odabire opciju dodavanja
				lokacije, a poslužitelj mu daje pristup jer je registriran te korisnik unosi podatke o lokaciji (ime lokacije, kategoriju te ocjenu/oznaku prikladnosti). Poslužitelj 
				za unesene podatke o lokaciji dohvaća iz baze podataka podatke o toj lokaciji. Ako oni ne postoje u bazi podataka, spremaju se u bazu podataka te poslužitelj dohvaća 
				iz nje novi marker. Preuzima ga Google Maps API koji ga dodaje na kartu te poslužitelj dohvaća kartu s dodanim novim markerom i prikazuje je korisniku. Ako podaci o
				lokaciji već postoje u bazi, korisnik prima informaciju o postojanju podataka za željenu lokaciju.
				
				\begin{figure}[H]
					\includegraphics[scale=0.43]{slike/SekvencijskiDijagram2.png}
					\centering
					\caption{Sekvencijski dijagram za UC14}
					\label{fig:promjene}
				\end{figure}
				\eject
	
		\section{Ostali zahtjevi}
		
			\begin{packed_item}
	
        	\item  Sustav treba omogučiti rad više korisnika u stvarnom vremenu
        	\item  Korisničko sučelje i sustav moraju podržavati hrvatsku abecedu (dijakritičke znakove) pri unosu i prikazu tekstualnog sadržaja
        	\item  Izvršavanje dijela programa u kojem se pristupa bazi podataka ne smije trajati duže od nekoliko sekundi
            \item  Sustav treba biti implementiran kao web aplikacija koristeći objektni-orijentirane jezike
            \item Neispravno korištenje korisničkog sučelja ne smije narušiti funkcionalnost i rad sustava
            \item Sustav treba biti jednostavan za korištenje, korisnici se moraju znati koristiti sučeljem bez opširnih uputa
            \item Nadogradnja sustava ne smije narušavati postojeće funkcionalnosti sustava
            \item Sustav kao valutu koristi HRK
            \item Veza s bazom podataka mora biti kvalitetno zaštičena, brza i otporna na vanjske greške
            \item Pristup sustavu mora bizi omogućen iz javne mreže pomoću HTTPS
            \end{packed_item}
			 
			 
			 
	