\chapter{Arhitektura i dizajn sustava}
            \noindent Arhitektura se može podijeliti na četiri podsustava:
            \begin{packed_item}
	
        	\item  Web poslužitelj
        	\item  Web aplikacija
        	\item  Baza podataka
            \item  Servis za autentifikaciju
            \end{packed_item}
            \\
            
\textit{Web preglednik} je program koji korisniku omogućuje pregled web-stranica i multimedijalnih sadržaja vezanih uz njih. Svaki internetski preglednik je prevoditelj. Dakle, stranica je pisana u kodu koji preglednik nakon toga interpretira kao nešto svakome razumljivo. Korisnik putem web preglednika šalje zahtjev web poslužitelju.
\begin{figure}[H]
	\includegraphics[scale=0.4]{slike/ArhitekturaSustava.png} 
	\centering
	\caption{Prikaz arhitekture}
	\label{fig:promjene}
\end{figure}

\newline
\\
\textit{Web poslužitelj} osnova je rada web aplikacije. Njegova primarna zadaća je komunikacija klijent s aplikacijom. Komunikacija se odvija preko HTTP (engl. \textit{Hyper Text Transfer Protocol}) protokola, što je protokol u prijenosu informacija na webu. Poslužitelj je onaj koji pokreće web aplikaciju te joj prosljeđuje zahtjeve.
\newline
\\
Korisnik korisi \textit{web aplikaciju} za obrađivanje željenih zahtjeva. Web aplikacija obrađuje zahtjev te ovisno o zahtjevu, pristupa bazi podataka i/ili servisu za autentifikaciju nakon čega preko poslužitelja vraća korisniku odgovor u obliku HTML dokumenta vidljivog u web pregledniku.
\newline
\\
Programski jezik kojeg smo odabrali za izradu naše web aplikacije je JavaScript za koji koristimo React biblioteku (engl. \textit{library}) te Next.js radni okvir (engl. \textit{framework}. Za autentifikaciju koristimo Google-ov modul za autentifikaciju iz njihovog servisa Firebase. Odabrano razvojno okruženje je Microsoft Visual Studio Code. Iako ne potpuno, arhitektura sustava temeljit će se na MVC (Model-View-Controller) konceptu. MVC koncept nije potpuno podržan od strane Next.js-a jer je to radni okvir koji omogućava posluživanje React aplikacija sa servera, a u samom React-u MVC koncept nije potpuno podržan. No, radi lakše izrade i organizacije projekta, odlučili smo na što bliži način implementirati MVC koncept.
\newline
\\
Karakteristika MVC koncepta je nezavisan razvoj pojedinih dijelova aplikaicje što za posljedicu ima jednostasvnije ispitivanje kao i jednostsavno razvijanje i dodavanje novih svojstava u sustav.
\newline
\\
MVC koncept sastoji se od:

\begin{packed_item}
	
        	\item  \textbf{Model} - Središnja komponenta sustava. Predstavlja dinamičke strukture podataka, neovisne o korisničkom sučelju. Izravno upravlja podacima, logikom i pravilima aplikacije. Također prima ulazne podatke od Controllera
        	\item \textbf{View} - Bilo kakav prikaz podataka, poput grafa. Mogući su različiti prikazi iste informacije poput grafičkog ili tabličnog prikaza podataka.
        	\item  \textbf{Controller} - Prima ulaze i prilagođava ih za prosljeđivanje Modelu i Viewu. Upravlja korisničkim zahtjevima i zemeljem njih izvodi daljnju interakciju s ostalim elementima sustava.
            \end{packed_item}


	
		

		

				
		\section{Baza podataka}

  Za potrebe našeg sustava koristit ćemo Firestore, Google-ovu NoSQL cloud bazu podataka iz njihovog servisa Firebase koja svojom strukturom omogućava lake izmijene koje će pratiti skaliranje aplikacije. Gradivna jedinka baze je kolekcija (engl. \textit{collection}) definirana svojim imenom i skupom dokumenata (engl. \textit{documents}). Zadaća baze podataka je brza i jednostavna pohrana, izmjena i dohvat podataka za daljnju obradu. Baza podataka ove aplikacije sastoji se od sljedećih kolekcija:
  \begin{packed_item}
	
        	\item  users
        	\item  companies
        	\item  locations
            \end{packed_item}
		
			\subsection{Opis tablica}
			
				
				
				\textbf{Users}\hspace{1cm}  Ova tablica sadrži sve relevantne informacije za korisnike aplikacije. Sadrži atribute: UserId, Username, email i CompanyOwner(boolean atribut koji nam govori je li korisnik vlasnik obrta). Tablica je u vezi \textit{One-to-One} s tablicom Companies i u vezi \textit{One-to-Many} je s tablicom Locations. 

                
                \begin{longtblr}[
					label=none,
					entry=none
					]{
						width = \textwidth,
						colspec={|X[7,l]|X[6, l]|X[20, l]|}, 
						rowhead = 1,
					} %definicija širine tablice, širine stupaca, poravnanje i broja redaka naslova tablice
					\hline {\textbf{Users}}	 \\ \hline[3pt]
					\SetCell{LightGreen}UserId & VARCHAR	&  	Hashirani string jedinstven za svakog korisnika  	\\ \hline
					Username	& VARCHAR & Nadimak pod kojim će se korisnik predstavljati u aplikaciji  	\\ \hline 
					email & VARCHAR &  E-mail adresa korisnika \\ \hline 
					CompanyOwner & BOOLEAN	&  True ako je korisnik vlasnik obrta, inače False		\\ \hline  
				\end{longtblr}
                \textbf{Locations} \hspace{1cm} Ova tablica sadrži informacije o lokacijama dodanih od strane korisnika. Sadrži atribute: Ime lokacije, id korisnika koji je dodao lokaciju, status prikladnosti i koordinate. U vezi  \textit{Many-to-One} je s tablicom Users.
                \begin{longtblr}[
					label=none,
					entry=none
					]{
						width = \textwidth,
						colspec={|X[6,l]|X[6, l]|X[20, l]|}, 
						rowhead = 1,
					} %definicija širine tablice, širine stupaca, poravnanje i broja redaka naslova tablice
					\hline {\textbf{Locations}}	 \\ \hline[3pt]
					\SetCell{LightGreen}Name & VARCHAR	&  Ime lokacije  	\\ \hline
					\SetCell{LightBlue} UserId	& VARCHAR &  Hashirani string niz jedinstven za korisnika koji je dodao lokaciju 	\\ \hline  
					Prikladna & BOOLEAN	&  True ako je prikladna za ljubimce, inače False		\\ \hline 
                    Coordinates & VARCHAR &  Koordinate lokacije \\ \hline 
				\end{longtblr}
                \textbf{Companies} \hspace{1cm} Ova tablica sadrži informacije o dodanim obrtima. Posjeduje atribute: OIB obrta, Id vlasnika obrta, adresu, naziv obrta, kojoj kategoriji pripada i kratki opis obrta.  U vezi  \textit{One-to-One} je s tablicom Users.
                \begin{longtblr}[
					label=none,
					entry=none
					]{
						width = \textwidth,
						colspec={|X[6,l]|X[6, l]|X[20, l]|}, 
						rowhead = 1,
					} %definicija širine tablice, širine stupaca, poravnanje i broja redaka naslova tablice
					\hline {\textbf{Companies}}	 \\ \hline[3pt]
					\SetCell{LightGreen}OIB & VARCHAR	&  Niz od 11 znamenki karakterističan za tu pravnu osobu  	\\ \hline
					\SetCell{LightBlue} OwnerId	& VARCHAR &  Hashirani string niz jedinstven za korisnika koji je ujedino i vlasnik obrta 	\\ \hline  
					Adress & VARCHAR	&  Adresa obrta	\\ \hline 
                    Name & VARCHAR &  Ime obrta \\ \hline 
                    Type & VARCHAR &  Kategorija kojoj obrt pripada \\ \hline 
                    Phone & VARCHAR &  Kontakt broj obrta\\ \hline 
					Description & VARCHAR &  Kratki opis obrta \\ \hline 
				\end{longtblr}

                
				
				
			
			\subsection{Dijagram baze podataka}
				%\textit{ U ovom potpoglavlju potrebno je umetnuti dijagram baze podataka. Primarni i strani ključevi moraju biti označeni, a tablice povezane. Bazu podataka je potrebno normalizirati. Podsjetite se kolegija "Baze podataka".}
                %oznaci kljuceve
				Na dijagramu ključevi su "boldani", a strani ključevi podcrtani.
                \begin{figure}[H]
			\includegraphics[scale=1.2]{slike/Baza.png}
			\centering
			\caption{Dijagram baze podataka}
			\label{fig:promjene}
		          \end{figure}
			
			\eject
			
		\section{Dijagram razreda}
		
			
			
			Na slici 4.3 prikazan je dijagram razreda cijelog projekta. Implementirane metode direktno komuniciraju s bazom podataka te vraćaju tražene podatke.
			
			PrivatnaForm služi za unos informacija o privatnom korisniku, vrši provjeru ispravnosti unosa i ako je sve u redu kreira korisnika u firebaseu
			
			VlasnikForm služi za unos informacija o obrtu. Sadrži podatke potrebne za neki obrt, provjerava ispravnost unesenih podataka i ako je sve u redu kreira korisnika u firebaseu.
	
	        Firebase služi za konfiguraciju baze i komunikaciju s istom. 
	        
	        Context stvara kontekst za trenutnog korisnika
	        
	        UseMyHooks provjerava nalazi li se lozinka u "password blacklisti" i ako da, traži promjenu lozinke.
	        
	        Login korisniku prikazuje stranicu za ulogiravanje u aplikaciju. Prilikom log-ina gleda se postoji li korisnik u bazi, ako ne javlja se greška. Ako postoji ulogirava se u aplikaciju. Ne dozvoljava se upis dok sva polja nisu ispravno upisana.
	        
	        Register korisniku prikazuje stranicu za registraciju. U zavisnisti od odabranog prikazuje se ili "PrivatnaForma" ili "VlasnikForma"
	        
	        UserInfo prikazuje stranicu sa podatcima o ulogiranom korisniku. Prvotno funkcija povlači podatke o korisniku iz baze i ako je korisnik vlasnik firme, povlači i podatke o firmi. Te se isti podatci prikazuju na stranici
	        
	        \begin{figure}[H]
			\includegraphics[scale=0.2]{slike/DijagramRazreda.png}
			\centering
			\caption{Modeli iz Baze}
			\label{fig:promjene}
		          \end{figure}
			
			%\textbf{\textit{dio 2. revizije}}\\			
			
			%\textit{Prilikom druge predaje projekta dijagram razreda i opisi moraju odgovarati stvarnom stanju implementacije}
			
			
			
			%\eject
		
		%\section{Dijagram stanja}
			
			
			%\textbf{\textit{dio 2. revizije}}\\
			
			%\textit{Potrebno je priložiti dijagram stanja i opisati ga. Dovoljan je jedan dijagram stanja koji prikazuje \textbf{značajan dio funkcionalnosti} sustava. Na primjer, stanja korisničkog sučelja i tijek korištenja neke ključne funkcionalnosti jesu značajan dio sustava, a registracija i prijava nisu. }
			
			
			%\eject 
		
		%\section{Dijagram aktivnosti}
			
			%\textbf{\textit{dio 2. revizije}}\\
			
			% \textit{Potrebno je priložiti dijagram aktivnosti s pripadajućim opisom. Dijagram aktivnosti treba prikazivati značajan dio sustava.}
			
			%\eject
		%\section{Dijagram komponenti}
		
			%\textbf{\textit{dio 2. revizije}}\\
		
			 %\textit{Potrebno je priložiti dijagram komponenti s pripadajućim opisom. Dijagram komponenti treba prikazivati strukturu cijele aplikacije.}