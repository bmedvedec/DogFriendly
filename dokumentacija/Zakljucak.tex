\chapter{Zaključak i budući rad}
		%komentar
		\textbf{\textit{dio 2. revizije}}\\
		
		 %$\textit{U ovom poglavlju potrebno je napisati osvrt na vrijeme izrade projektnog zadatka, koji su tehnički izazovi prepoznati, jesu li riješeni ili kako bi mogli biti riješeni, koja su znanja stečena pri izradi projekta, koja bi znanja bila posebno potrebna za brže i kvalitetnije ostvarenje projekta i koje bi bile perspektive za nastavak rada u projektnoj grupi.}
		
		% \textit{Potrebno je točno popisati funkcionalnosti koje nisu implementirane u ostvarenoj aplikaciji.}

        Zadatak naše grupe bio je razvoj web aplikacije za prikaz, lociranje i dodavanje lokacija pogodnih za pse. Nakon 17 tjedana rada u timu i razvoja, ostvarili smo zadani cilj. Sama provedba projekta bila je kroz dvije faze.

        Prva faza projekta uključivala je okupljanje tima za razvoj aplikacije, dodjelu projektnog zadatka i rad na dokumentiranju zahtjeva. Kvalitetan rad na prvom dijelu uvelike nam olakšava daljnji rad. 
        
        Druga faza projekta bila je puno intenzivnija po pitanju samostalnog rada članova. Osim realizacije rješenja, u drugoj fazi je bilo potrebno dokumentirati ostale UML dijagrame i izraditi popratnu dokumentaciju kako budući korisnici mogli lakše korisititi ili vršiti preinake na sustavu. Dobro izrađen kostur uštedio nam je puno vremena prilikom izrade aplikacije.
        
        Komunikacija među članovima tima bila je putem Discorda čime smo postigli informiranost i uključenost svih članova grupe. Moguće proširenje postojeće inačice sustava je izrada mobilne aplikacije čime bi cilj projektnog zadatka bio ostvaren u većoj mjeri no s web aplikacijom.
        Sudjelovanje na ovakvom projektu bilo je vrijedno iskustvo svim članovima tima jer smo kroz intenzivnih nekoliko tjedana rada iskusili zajednički rad na istom projektu, te naučili korisitit nove tehnologije. Također, osjetili smo važnost dobre vremenske organiziranosti i koordiniranosti između članova tima. Zadovoljni smo postignutim rezultatom bez obzira na prostor za usavrđavanje izrađene aplikacije.
		
		\eject 