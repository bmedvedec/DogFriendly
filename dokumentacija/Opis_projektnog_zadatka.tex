\chapter{Opis projektnog zadatka}
		
		Cilj ovog projekta je razviti programsku podršku za stvaranje web aplikacije Dog Friendly koja će korisniku pomoći pronaći lokacije koje su prikladne i dostojne čovjekovog najboljeg prijatelja. Ova aplikacija  će vlasnicima i ljubiteljima pasa omogućiti pregled prikladnih i neprikladnih lokacija na interaktivnoj karti i time olakšati  druženje, igru i odmor.
		Aplikacija  pruža i dodatak novih lokacija na interaktivnu kartu te za svaku novu lokaciju potrebno je dodatni nekoliko ključnih informacija poput imena lokacije, rating i kategorija( parkić, plaža, dućan, kafić, restoran, veterinarska
		ambulanta, frizerski salon, itd.). Ukoliko pritisnemo na indikator lokacije prikazat će nam se opis lokacije i kontakt. 
		Korisnike ove aplikacije dijelimo na tri vrste: neregistrirani korisnici, registrirani korisnici. Neregistriranim korisnicima pružene su osnovne funkcionalnosti aplikacije(pregled interaktivne karte). Registrirani korisnici su korisnici koji su prošli kroz proces registracije koji se sastoji od 2 koraka. U prvom koraku korisnik ispunjava formu s korisničkim imenom, e-mail adresom i zaporukom. Nakon ispunjavanja korisnik na svoju e-mail adresu prima obavijest o registraciji i traži ga se da potvrdi svoju registraciju. Nakon potvrde proces registracije je završen. Ukoliko korisnik zaboravi svoju zaporuku ili je dobio ideju za bolje korisničko ime pruža mu se promjena oba korisnička podatka. Registriranim korisnicima pružene su sve funkcionalnisti aplikacije koje imaju i neregistrirani korisnici uz priliku dodavanja novih lokacija na kartu.
		\eject
		
	